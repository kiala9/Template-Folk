%%
% Copyright (c) 2017 - 2025, Pascal Wagler;
% Copyright (c) 2014 - 2025, John MacFarlane
%
% All rights reserved.
%
% Redistribution and use in source and binary forms, with or without
% modification, are permitted provided that the following conditions
% are met:
%
% - Redistributions of source code must retain the above copyright
% notice, this list of conditions and the following disclaimer.
%
% - Redistributions in binary form must reproduce the above copyright
% notice, this list of conditions and the following disclaimer in the
% documentation and/or other materials provided with the distribution.
%
% - Neither the name of John MacFarlane nor the names of other
% contributors may be used to endorse or promote products derived
% from this software without specific prior written permission.
%
% THIS SOFTWARE IS PROVIDED BY THE COPYRIGHT HOLDERS AND CONTRIBUTORS
% "AS IS" AND ANY EXPRESS OR IMPLIED WARRANTIES, INCLUDING, BUT NOT
% LIMITED TO, THE IMPLIED WARRANTIES OF MERCHANTABILITY AND FITNESS
% FOR A PARTICULAR PURPOSE ARE DISCLAIMED. IN NO EVENT SHALL THE
% COPYRIGHT OWNER OR CONTRIBUTORS BE LIABLE FOR ANY DIRECT, INDIRECT,
% INCIDENTAL, SPECIAL, EXEMPLARY, OR CONSEQUENTIAL DAMAGES (INCLUDING,
% BUT NOT LIMITED TO, PROCUREMENT OF SUBSTITUTE GOODS OR SERVICES;
% LOSS OF USE, DATA, OR PROFITS; OR BUSINESS INTERRUPTION) HOWEVER
% CAUSED AND ON ANY THEORY OF LIABILITY, WHETHER IN CONTRACT, STRICT
% LIABILITY, OR TORT (INCLUDING NEGLIGENCE OR OTHERWISE) ARISING IN
% ANY WAY OUT OF THE USE OF THIS SOFTWARE, EVEN IF ADVISED OF THE
% POSSIBILITY OF SUCH DAMAGE.
%%

%%
% This is the Eisvogel pandoc LaTeX template.
%
% For usage information and examples visit the official GitHub page:
% https://github.com/Wandmalfarbe/pandoc-latex-template
%%
% Options for packages loaded elsewhere
\PassOptionsToPackage{unicode}{hyperref}
\PassOptionsToPackage{hyphens}{url}
\PassOptionsToPackage{dvipsnames,svgnames,x11names,table}{xcolor}
\PassOptionsToPackage{space}{xeCJK}
\documentclass[
  ignorenonframetext,
  chinese,
]{beamer}
\newif\ifbibliography
\usepackage{pgfpages}
\setbeamertemplate{caption}[numbered]
\setbeamertemplate{caption label separator}{: }
\setbeamercolor{caption name}{fg=normal text.fg}
\beamertemplatenavigationsymbolsempty
% remove section numbering
\setbeamertemplate{part page}{
  \centering
  \begin{beamercolorbox}[sep=16pt,center]{part title}
    \usebeamerfont{part title}\insertpart\par
  \end{beamercolorbox}
}
\setbeamertemplate{section page}{
  \centering
  \begin{beamercolorbox}[sep=12pt,center]{section title}
    \usebeamerfont{section title}\insertsection\par
  \end{beamercolorbox}
}
\setbeamertemplate{subsection page}{
  \centering
  \begin{beamercolorbox}[sep=8pt,center]{subsection title}
    \usebeamerfont{subsection title}\insertsubsection\par
  \end{beamercolorbox}
}
% Prevent slide breaks in the middle of a paragraph
\widowpenalties 1 10000
\raggedbottom
\AtBeginPart{
  \frame{\partpage}
}
\AtBeginSection{
  \ifbibliography
  \else
    \frame{\sectionpage}
  \fi
}
\AtBeginSubsection{
  \frame{\subsectionpage}
}
\usepackage{iftex}
\ifPDFTeX
  \usepackage[T1]{fontenc}
  \usepackage[utf8]{inputenc}
  \usepackage{textcomp} % provide euro and other symbols
\else % if luatex or xetex
  \usepackage{unicode-math} % this also loads fontspec
  \defaultfontfeatures{Scale=MatchLowercase}
  \defaultfontfeatures[\rmfamily]{Ligatures=TeX,Scale=1}
\fi
\usepackage{lmodern}
\ifPDFTeX\else
  % xetex/luatex font selection
  \ifXeTeX
    \usepackage{xeCJK}
    \setCJKmainfont[]{Microsoft YaHei}
          \fi
  \ifLuaTeX
    \usepackage[]{luatexja-fontspec}
    \setmainjfont[]{Microsoft YaHei}
  \fi
\fi
% Use upquote if available, for straight quotes in verbatim environments
\IfFileExists{upquote.sty}{\usepackage{upquote}}{}
\IfFileExists{microtype.sty}{% use microtype if available
  \usepackage[]{microtype}
  \UseMicrotypeSet[protrusion]{basicmath} % disable protrusion for tt fonts
}{}
% Use setspace anyway because we change the default line spacing.
% The spacing is changed early to affect the titlepage and the TOC.
\usepackage{setspace}
\setstretch{1.2}
\makeatletter
\@ifundefined{KOMAClassName}{% if non-KOMA class
  \IfFileExists{parskip.sty}{%
    \usepackage{parskip}
  }{% else
    \setlength{\parindent}{0pt}
    \setlength{\parskip}{6pt plus 2pt minus 1pt}}
}{% if KOMA class
  \KOMAoptions{parskip=half}}
\makeatother
\usepackage{listings}
\newcommand{\passthrough}[1]{#1}
\lstset{defaultdialect=[5.3]Lua}
\lstset{defaultdialect=[x86masm]Assembler}
\ifLuaTeX
\usepackage[bidi=basic]{babel}
\else
\usepackage[bidi=default]{babel}
\fi
\babelprovide[main,import]{chinese}
% get rid of language-specific shorthands (see #6817):
\let\LanguageShortHands\languageshorthands
\def\languageshorthands#1{}
\setlength{\emergencystretch}{3em} % prevent overfull lines
\providecommand{\tightlist}{%
  \setlength{\itemsep}{0pt}\setlength{\parskip}{0pt}}
\usepackage{bookmark}
\IfFileExists{xurl.sty}{\usepackage{xurl}}{} % add URL line breaks if available
\urlstyle{same}
\definecolor{default-linkcolor}{HTML}{A50000}
\definecolor{default-filecolor}{HTML}{A50000}
\definecolor{default-citecolor}{HTML}{4077C0}
\definecolor{default-urlcolor}{HTML}{4077C0}

\hypersetup{
  pdftitle={FPGA与CPLD架构基础},
  pdfauthor={周贤中},
  pdflang={zh-cn},
  pdfsubject={Markdown},
  pdfkeywords={FPGA, 系统设计},
  hidelinks,
  breaklinks=true,
  pdfcreator={LaTeX via pandoc with the Eisvogel template}}
\title{FPGA与CPLD架构基础}
\author{周贤中}
\date{2025-02-22}

\begin{document}
\frame{\titlepage}

\section{前言}\label{ux524dux8a00}

\begin{frame}{\textbf{PLD的基本概念}}
\phantomsection\label{pldux7684ux57faux672cux6982ux5ff5}
可编程逻辑器件(Programmable Logic Device,
PLD)是一类可以通过编程实现特定逻辑功能的集成电路。PLD的主要特点是通过软件工具对硬件进行配置,从而实现不同的逻辑功能。以下是PLD的核心概念:
\end{frame}

\begin{frame}
\begin{block}{\textbf{PLD的分类}}
\phantomsection\label{pldux7684ux5206ux7c7b}
PLD根据其复杂性和功能可以分为以下几类:

\begin{enumerate}
\tightlist
\item
  \textbf{SPLD(简单可编程逻辑器件)}:

  \begin{itemize}
  \tightlist
  \item
    包括PAL(可编程阵列逻辑)和GAL(通用阵列逻辑)。\\
  \item
    适合实现简单的逻辑功能,如组合逻辑和时序逻辑。\\
  \end{itemize}
\item
  \textbf{CPLD(复杂可编程逻辑器件)}:

  \begin{itemize}
  \tightlist
  \item
    由多个SPLD模块和可编程互连资源组成。\\
  \item
    适合实现中等复杂度的逻辑功能,如状态机和接口转换。\\
  \end{itemize}
\item
  \textbf{FPGA(现场可编程门阵列)}:

  \begin{itemize}
  \tightlist
  \item
    由大量可配置逻辑块(CLB)、可编程互连资源和I/O单元组成。\\
  \item
    适合实现复杂的逻辑功能,如数字信号处理和通信协议。
  \end{itemize}
\end{enumerate}
\end{block}
\end{frame}

\begin{frame}
\begin{block}{\textbf{PLD的特点}}
\phantomsection\label{pldux7684ux7279ux70b9}
\begin{itemize}
\tightlist
\item
  \textbf{可编程性}: 用户可以通过编程实现不同的逻辑功能。\\
\item
  \textbf{灵活性}: 支持动态重构,适合快速原型设计和迭代开发。\\
\item
  \textbf{开发周期短}: 相比ASIC,PLD的开发周期显著缩短。\\
\item
  \textbf{成本适中}: 适合中小批量生产,无需高昂的流片成本。
\end{itemize}
\end{block}
\end{frame}

\begin{frame}
\begin{block}{\textbf{PLD的应用场景}}
\phantomsection\label{pldux7684ux5e94ux7528ux573aux666f}
\begin{itemize}
\tightlist
\item
  通信设备(如5G基站、网络交换机)\\
\item
  工业控制(如PLC、机器人控制)\\
\item
  消费电子(如视频处理、游戏硬件)\\
\item
  原型设计与验证(如ASIC设计的前期验证)
\end{itemize}
\end{block}
\end{frame}

\begin{frame}{\textbf{ASIC的基本概念}}
\phantomsection\label{asicux7684ux57faux672cux6982ux5ff5}
专用集成电路(Application-Specific Integrated Circuit,
ASIC)是为特定应用定制的集成电路,设计完成后功能固定,无法更改。以下是ASIC的核心概念:
\end{frame}

\begin{frame}
\begin{block}{\textbf{ASIC的分类}}
\phantomsection\label{asicux7684ux5206ux7c7b}
\begin{enumerate}
\tightlist
\item
  \textbf{全定制ASIC}:

  \begin{itemize}
  \tightlist
  \item
    从晶体管级别进行设计,性能最优,但开发成本和时间最高。\\
  \end{itemize}
\item
  \textbf{半定制ASIC}:

  \begin{itemize}
  \tightlist
  \item
    基于标准单元库或门阵列进行设计,性能和成本介于全定制和可编程器件之间。\\
  \end{itemize}
\item
  \textbf{结构化ASIC}:

  \begin{itemize}
  \tightlist
  \item
    基于预定义的硬件结构进行设计,开发周期较短,成本较低。
  \end{itemize}
\end{enumerate}
\end{block}
\end{frame}

\begin{frame}
\begin{block}{\textbf{ASIC的特点}}
\phantomsection\label{asicux7684ux7279ux70b9}
\begin{itemize}
\tightlist
\item
  \textbf{高性能}: 针对特定应用优化,性能优于PLD。\\
\item
  \textbf{低功耗}: 针对特定应用进行功耗优化,功耗低于PLD。\\
\item
  \textbf{高成本}: 开发成本高,适合大批量生产。\\
\item
  \textbf{开发周期长}: 从设计到流片需要数月甚至数年时间。
\end{itemize}
\end{block}
\end{frame}

\begin{frame}
\begin{block}{\textbf{ASIC的应用场景}}
\phantomsection\label{asicux7684ux5e94ux7528ux573aux666f}
\begin{itemize}
\tightlist
\item
  消费电子(如智能手机、平板电脑)\\
\item
  汽车电子(如ADAS、车载娱乐系统)\\
\item
  数据中心(如AI加速芯片、网络处理器)\\
\item
  工业设备(如传感器、控制器)
\end{itemize}
\end{block}
\end{frame}

\begin{frame}
\begin{block}{PLD与ASIC的详细对比分析}
\phantomsection\label{pldux4e0easicux7684ux8be6ux7ec6ux5bf9ux6bd4ux5206ux6790}
\textbf{设计灵活性}

\begin{itemize}
\tightlist
\item
  \textbf{PLD}:

  \begin{itemize}
  \tightlist
  \item
    高灵活性,可编程实现任意逻辑功能\\
  \item
    支持动态重构\\
  \item
    适合快速原型设计和迭代开发\\
  \item
    适用于需频繁更新的场景\\
  \end{itemize}
\item
  \textbf{ASIC}:

  \begin{itemize}
  \tightlist
  \item
    功能固化成品后无法修改\\
  \item
    适合固定功能场景\\
  \item
    设计修改需重新流片,成本高昂
  \end{itemize}
\end{itemize}
\end{block}
\end{frame}

\begin{frame}
\textbf{性能}

\begin{itemize}
\tightlist
\item
  \textbf{PLD}:

  \begin{itemize}
  \tightlist
  \item
    中等复杂度设计适用\\
  \item
    信号传输延迟较高\\
  \item
    适合对性能要求不高的场景\\
  \end{itemize}
\item
  \textbf{ASIC}:

  \begin{itemize}
  \tightlist
  \item
    性能最高,针对特定场景优化\\
  \item
    信号传输延迟低\\
  \item
    适合高速通信、AI加速等高性能需求场景
  \end{itemize}
\end{itemize}
\end{frame}

\begin{frame}
\textbf{功耗}

\begin{itemize}
\tightlist
\item
  \textbf{PLD}:

  \begin{itemize}
  \tightlist
  \item
    功耗较高(静态/动态)\\
  \item
    适合对功耗要求宽松的场景\\
  \end{itemize}
\item
  \textbf{ASIC}:

  \begin{itemize}
  \tightlist
  \item
    功耗最低,经定制优化\\
  \item
    适合移动设备、物联网等低功耗场景
  \end{itemize}
\end{itemize}
\end{frame}

\begin{frame}
\textbf{成本}

\begin{itemize}
\tightlist
\item
  \textbf{PLD}:

  \begin{itemize}
  \tightlist
  \item
    中等成本,中小批量生产适用\\
  \item
    开发成本低,无流片费用\\
  \end{itemize}
\item
  \textbf{ASIC}:

  \begin{itemize}
  \tightlist
  \item
    大批量时单位成本低\\
  \item
    流片成本高昂\\
  \item
    适合大规模生产分摊成本
  \end{itemize}
\end{itemize}
\end{frame}

\begin{frame}
\textbf{开发周期}

\begin{itemize}
\tightlist
\item
  \textbf{PLD}:

  \begin{itemize}
  \tightlist
  \item
    周期短(数周至数月)\\
  \item
    设计流程简单\\
  \item
    适合时间紧迫项目\\
  \end{itemize}
\item
  \textbf{ASIC}:

  \begin{itemize}
  \tightlist
  \item
    周期长(数月至数年)\\
  \item
    需复杂物理设计和验证流程\\
  \item
    适合长期高要求项目
  \end{itemize}
\end{itemize}
\end{frame}

\begin{frame}
\textbf{开发工具}

\begin{itemize}
\tightlist
\item
  \textbf{PLD}:

  \begin{itemize}
  \tightlist
  \item
    需专用EDA工具(如Vivado, Quartus)\\
  \item
    工具链成熟,快速验证\\
  \item
    适合PLD工程师\\
  \end{itemize}
\item
  \textbf{ASIC}:

  \begin{itemize}
  \tightlist
  \item
    需ASIC工具链(如Cadence, Synopsys)\\
  \item
    工具复杂,需专业团队\\
  \item
    适用经验丰富的ASIC工程师
  \end{itemize}
\end{itemize}
\end{frame}

\begin{frame}
\textbf{适用场景}

\begin{itemize}
\tightlist
\item
  \textbf{PLD}:

  \begin{itemize}
  \tightlist
  \item
    通信设备(5G基站、交换机)\\
  \item
    工业控制(PLC、机器人)\\
  \item
    消费电子(视频处理)\\
  \item
    ASIC原型验证\\
  \end{itemize}
\item
  \textbf{ASIC}:

  \begin{itemize}
  \tightlist
  \item
    消费电子(手机、平板)\\
  \item
    汽车电子(ADAS)\\
  \item
    数据中心(AI加速器)\\
  \item
    工业设备(传感器)
  \end{itemize}
\end{itemize}
\end{frame}

\begin{frame}
\textbf{量产成本}

\begin{itemize}
\tightlist
\item
  \textbf{PLD}:

  \begin{itemize}
  \tightlist
  \item
    中小批量成本较高\\
  \item
    单位成本随规模降低\\
  \end{itemize}
\item
  \textbf{ASIC}:

  \begin{itemize}
  \tightlist
  \item
    大批量单位成本极低\\
  \item
    适合大规模生产
  \end{itemize}
\end{itemize}
\end{frame}

\begin{frame}
\textbf{可编程性}

\begin{itemize}
\tightlist
\item
  \textbf{PLD}:

  \begin{itemize}
  \tightlist
  \item
    可重复编程和动态重构\\
  \item
    支持功能灵活调整\\
  \end{itemize}
\item
  \textbf{ASIC}:

  \begin{itemize}
  \tightlist
  \item
    功能固化成形后不可修改
  \end{itemize}
\end{itemize}
\end{frame}

\begin{frame}
\textbf{集成度}

\begin{itemize}
\tightlist
\item
  \textbf{PLD}:

  \begin{itemize}
  \tightlist
  \item
    高集成度(数百万逻辑单元)\\
  \item
    支持复杂算法和协议\\
  \end{itemize}
\item
  \textbf{ASIC}:

  \begin{itemize}
  \tightlist
  \item
    定制化集成更多模块\\
  \item
    优化集成度和性能
  \end{itemize}
\end{itemize}

\textbf{设计复杂度}

\begin{itemize}
\tightlist
\item
  \textbf{PLD}:

  \begin{itemize}
  \tightlist
  \item
    中等复杂度,适用中小规模设计\\
  \item
    工具提供丰富IP核\\
  \end{itemize}
\item
  \textbf{ASIC}:

  \begin{itemize}
  \tightlist
  \item
    复杂度高,需完整流程(逻辑设计→物理设计→流片)\\
  \item
    需大型专业团队
  \end{itemize}
\end{itemize}
\end{frame}

\begin{frame}
\textbf{风险与可靠性}

\begin{itemize}
\tightlist
\item
  \textbf{PLD}:

  \begin{itemize}
  \tightlist
  \item
    风险低(可重新编程修复)\\
  \item
    适合原型验证\\
  \item
    可靠性依赖器件寿命\\
  \end{itemize}
\item
  \textbf{ASIC}:

  \begin{itemize}
  \tightlist
  \item
    风险高(需重新流片修复错误)\\
  \item
    需充分验证设计\\
  \item
    高可靠性(经应用优化)
  \end{itemize}
\end{itemize}
\end{frame}

\begin{frame}
\textbf{生态系统}

\begin{itemize}
\tightlist
\item
  \textbf{PLD}:

  \begin{itemize}
  \tightlist
  \item
    成熟生态(工具、IP核、社区)\\
  \item
    厂商支持完善(Xilinx/Intel)\\
  \item
    适合中小企业和初创\\
  \end{itemize}
\item
  \textbf{ASIC}:

  \begin{itemize}
  \tightlist
  \item
    需专业团队和代工厂(TSMC/三星)\\
  \item
    工具链复杂,成本高昂\\
  \item
    适合大企业或资金充足项目
  \end{itemize}
\end{itemize}
\end{frame}

\begin{frame}
\begin{block}{关键总结}
\phantomsection\label{ux5173ux952eux603bux7ed3}
\end{block}

\begin{block}{特性对比}
\phantomsection\label{ux7279ux6027ux5bf9ux6bd4}
\begin{itemize}
\tightlist
\item
  \textbf{灵活性与成本}:

  \begin{itemize}
  \tightlist
  \item
    \textbf{PLD优势场景}: 快速迭代、中小批量、功能待定
  \item
    \textbf{ASIC优势场景}: 功能固化、超大规模量产、高性能/低功耗需求
  \end{itemize}
\item
  \textbf{风险控制}:

  \begin{itemize}
  \tightlist
  \item
    \textbf{PLD优势场景}: 低风险原型验证、设计调试灵活
  \item
    \textbf{ASIC优势场景}: 高风险需前置验证、但量产可靠性极高
  \end{itemize}
\item
  \textbf{适用阶段}:

  \begin{itemize}
  \tightlist
  \item
    \textbf{PLD优势场景}: 前期开发验证、动态需求场景
  \item
    \textbf{ASIC优势场景}: 成熟方案固化、长期稳定量产
  \end{itemize}
\end{itemize}
\end{block}
\end{frame}

\begin{frame}
\begin{block}{生态系统}
\phantomsection\label{ux751fux6001ux7cfbux7edf}
PLD(可编程逻辑器件)和ASIC(专用集成电路)在生态系统方面的对比:

\textbf{生态系统成熟度}

\begin{itemize}
\tightlist
\item
  \textbf{PLD}: 成熟度高,形成了完整的工具链和社区支持体系。\\
\item
  \textbf{ASIC}: 复杂度高,需要专业供应链体系支撑。
\end{itemize}

\textbf{核心厂商}

\begin{itemize}
\tightlist
\item
  \textbf{PLD}:
  Xilinx(现隶属于AMD)、Intel(Altera系列)、莱迪思半导体等。\\
\item
  \textbf{ASIC}: 台积电(TSMC)、三星、GlobalFoundries等晶圆代工厂。
\end{itemize}
\end{block}
\end{frame}

\begin{frame}
\textbf{关键资源}

\begin{itemize}
\tightlist
\item
  \textbf{PLD}:

  \begin{itemize}
  \tightlist
  \item
    丰富的IP核库\\
  \item
    可视化开发工具链\\
  \item
    开发者社区和开源项目\\
  \end{itemize}
\item
  \textbf{ASIC}:

  \begin{itemize}
  \tightlist
  \item
    PDK(工艺设计套件)\\
  \item
    硅验证IP\\
  \item
    第三方设计服务(如ARM授权核)
  \end{itemize}
\end{itemize}
\end{frame}

\begin{frame}
\textbf{技术支持}

\begin{itemize}
\tightlist
\item
  \textbf{PLD}:

  \begin{itemize}
  \tightlist
  \item
    官方培训课程\\
  \item
    在线论坛和文档支持\\
  \item
    参考设计及应用手册\\
  \end{itemize}
\item
  \textbf{ASIC}:

  \begin{itemize}
  \tightlist
  \item
    需要付费技术支持协议\\
  \item
    依赖设计服务公司(如Cadence/Synopsys专业支持)
  \end{itemize}
\end{itemize}
\end{frame}

\begin{frame}
\textbf{准入门槛}

\begin{itemize}
\tightlist
\item
  \textbf{PLD}:

  \begin{itemize}
  \tightlist
  \item
    较低:开发板价格亲民(百至万元级),个人开发者可通过学习掌握基础开发。\\
  \end{itemize}
\item
  \textbf{ASIC}:

  \begin{itemize}
  \tightlist
  \item
    极高:需要数百万美元起投的流片费用,要求掌握完整的IC设计方法论。
  \end{itemize}
\end{itemize}
\end{frame}

\begin{frame}
\textbf{协作体系}

\begin{itemize}
\tightlist
\item
  \textbf{PLD}:

  \begin{itemize}
  \tightlist
  \item
    高校合作计划\\
  \item
    创客社区支持\\
  \item
    云平台开发环境(如AWS FPGA)\\
  \end{itemize}
\item
  \textbf{ASIC}:

  \begin{itemize}
  \tightlist
  \item
    EDA厂商联盟\\
  \item
    设计服务生态链(IP供应商→设计公司→封测厂)\\
  \item
    专业代工生态系统
  \end{itemize}
\end{itemize}
\end{frame}

\begin{frame}
\textbf{成本特征}

\begin{itemize}
\tightlist
\item
  \textbf{PLD}:

  \begin{itemize}
  \tightlist
  \item
    前期投入成本低\\
  \item
    按需购买开发许可\\
  \item
    硬件成本随型号递增\\
  \end{itemize}
\item
  \textbf{ASIC}:

  \begin{itemize}
  \tightlist
  \item
    NRE(非重复性工程)费用极高(百万美元级)\\
  \item
    需要长期维护成本\\
  \item
    MPW(多项目晶圆)可降低初期成本
  \end{itemize}
\end{itemize}
\end{frame}

\begin{frame}
\textbf{适用项目规模}

\begin{itemize}
\tightlist
\item
  \textbf{PLD}:

  \begin{itemize}
  \tightlist
  \item
    最适合中小团队验证项目、研究性课题和产品前期开发阶段。\\
  \end{itemize}
\item
  \textbf{ASIC}:

  \begin{itemize}
  \tightlist
  \item
    仅适合千万级预算项目、明确量产的成熟设计以及需长期维护迭代的旗舰产品。
  \end{itemize}
\end{itemize}
\end{frame}

\begin{frame}
\textbf{敏捷开发支持}

\begin{itemize}
\tightlist
\item
  \textbf{PLD}:

  \begin{itemize}
  \tightlist
  \item
    支持动态局部重配置、软硬件协同开发和快速迭代验证。\\
  \end{itemize}
\item
  \textbf{ASIC}:

  \begin{itemize}
  \tightlist
  \item
    仅支持硅前仿真验证,流片后无法修改,迭代周期以季度/年为单位计算。
  \end{itemize}
\end{itemize}
\end{frame}

\begin{frame}
\textbf{风险缓冲机制}

\begin{itemize}
\tightlist
\item
  \textbf{PLD}:

  \begin{itemize}
  \tightlist
  \item
    可重复编程降低出错成本\\
  \item
    支持硬件热修复\\
  \item
    多版本并行开发\\
  \end{itemize}
\item
  \textbf{ASIC}:

  \begin{itemize}
  \tightlist
  \item
    需通过FIB(聚焦离子束)修改芯片\\
  \item
    金属掩模费用高昂\\
  \item
    重大错误可能导致项目流产
  \end{itemize}
\end{itemize}
\end{frame}

\begin{frame}
\begin{block}{典型案例}
\phantomsection\label{ux5178ux578bux6848ux4f8b}
\begin{block}{PLD的典型应用}
\phantomsection\label{pldux7684ux5178ux578bux5e94ux7528}
\begin{itemize}
\tightlist
\item
  通信设备:
  5G基站、网络交换机等设备中,FPGA用于实现高速信号处理和协议转换。
\item
  工业控制:
  PLC(可编程逻辑控制器)和机器人控制中,CPLD用于实现逻辑控制和接口转换。
\item
  消费电子: 视频处理、游戏硬件等设备中,FPGA用于实现图像处理和算法加速。
\item
  原型设计与验证: 在ASIC设计的前期,FPGA用于功能验证和性能测试。
\end{itemize}
\end{block}
\end{block}
\end{frame}

\begin{frame}
\begin{block}{ASIC的典型应用}
\phantomsection\label{asicux7684ux5178ux578bux5e94ux7528}
\begin{itemize}
\tightlist
\item
  消费电子:
  智能手机、平板电脑等设备中,ASIC用于实现处理器、基带芯片和图像传感器。
\item
  汽车电子:
  ADAS(高级驾驶辅助系统)和车载娱乐系统中,ASIC用于实现传感器处理和控制算法。
\item
  数据中心: AI加速芯片和网络处理器中,ASIC用于实现高性能计算和数据传输。
\item
  工业设备: 传感器和控制器中,ASIC用于实现高精度测量和控制逻辑。
\end{itemize}
\end{block}
\end{frame}

\begin{frame}{\textbf{全球FPGA与CPLD市场概况}}
\phantomsection\label{ux5168ux7403fpgaux4e0ecpldux5e02ux573aux6982ux51b5}
FPGA和CPLD是可编程逻辑器件(PLD)的两大主要类别,广泛应用于通信、工业控制、消费电子、汽车电子、数据中心等领域。以下是两者的市场现状:

\begin{block}{\textbf{FPGA市场}}
\phantomsection\label{fpgaux5e02ux573a}
\begin{itemize}
\tightlist
\item
  \textbf{市场规模}: 根据市场研究机构的数据,2022年全球FPGA市场规模约为
  \textbf{80亿美元},预计到2027年将增长至
  \textbf{120亿美元},年均复合增长率(CAGR)约为 \textbf{8\%}。
\item
  \textbf{主要应用领域}:

  \begin{itemize}
  \tightlist
  \item
    通信(5G、基站、网络设备):占比约 \textbf{40\%}\\
  \item
    数据中心(AI加速、云计算):占比约 \textbf{25\%}\\
  \item
    工业控制(自动化、机器人):占比约 \textbf{15\%}\\
  \item
    汽车电子(ADAS、智能驾驶):占比约 \textbf{10\%}\\
  \item
    消费电子(视频处理、游戏):占比约 \textbf{10\%}\\
  \end{itemize}
\item
  \textbf{竞争格局}:

  \begin{itemize}
  \tightlist
  \item
    国际巨头主导:Xilinx(现为AMD旗下)和Intel(Altera)占据全球FPGA市场约
    \textbf{80\%}的份额。\\
  \item
    国内厂商崛起:紫光同创、复旦微电子、高云半导体等国内企业逐步扩大市场份额,但总体占比仍较低(约
    \textbf{5\%-10\%})。
  \end{itemize}
\end{itemize}
\end{block}
\end{frame}

\begin{frame}
\begin{block}{\textbf{CPLD市场}}
\phantomsection\label{cpldux5e02ux573a}
\begin{itemize}
\tightlist
\item
  \textbf{市场规模}: CPLD市场规模相对较小,2022年全球市场规模约为
  \textbf{10亿美元},预计到2027年将增长至
  \textbf{12亿美元},年均复合增长率(CAGR)约为 \textbf{3\%}。
\item
  \textbf{主要应用领域}:

  \begin{itemize}
  \tightlist
  \item
    工业控制(逻辑控制、接口转换):占比约 \textbf{50\%}\\
  \item
    消费电子(显示驱动、电源管理):占比约 \textbf{30\%}\\
  \item
    通信(协议转换、信号处理):占比约 \textbf{20\%}\\
  \end{itemize}
\item
  \textbf{竞争格局}:

  \begin{itemize}
  \tightlist
  \item
    国际厂商主导:Lattice
    Semiconductor和Microchip(收购Atmel)占据全球CPLD市场的主要份额。\\
  \item
    国内厂商参与:国内企业在CPLD领域的布局较少,市场份额较低。
  \end{itemize}
\end{itemize}
\end{block}
\end{frame}

\begin{frame}{\textbf{FPGA与CPLD的市场份额对比}}
\phantomsection\label{fpgaux4e0ecpldux7684ux5e02ux573aux4efdux989dux5bf9ux6bd4}
以下是FPGA与CPLD在全球市场中的份额对比:

在全球可编程逻辑器件市场中,FPGA和CPLD的市场表现存在明显差异:

\begin{itemize}
\tightlist
\item
  \textbf{FPGA}:

  \begin{itemize}
  \tightlist
  \item
    2022年市场规模:80亿美元
  \item
    2027年预测市场规模:120亿美元
  \item
    年均复合增长率(CAGR):8\%
  \item
    主要厂商:赛灵思(现隶属于AMD)、英特尔(Altera系列)、紫光同创、复旦微电子
  \end{itemize}
\item
  \textbf{CPLD}:

  \begin{itemize}
  \tightlist
  \item
    2022年市场规模:10亿美元
  \item
    2027年预测市场规模:12亿美元
  \item
    年均复合增长率(CAGR):3\%
  \item
    主要厂商:莱迪思半导体、微芯科技
  \end{itemize}
\end{itemize}
\end{frame}

\begin{frame}
FPGA的高增长主要得益于其在高性能计算、通信和人工智能等领域的广泛应用,而CPLD则因其处理能力有限,更多应用于复杂度较低的场景,市场增长相对较慢。
\end{frame}

\begin{frame}{\textbf{FPGA与CPLD的市场驱动因素}}
\phantomsection\label{fpgaux4e0ecpldux7684ux5e02ux573aux9a71ux52a8ux56e0ux7d20}
\begin{block}{\textbf{FPGA市场驱动因素}}
\phantomsection\label{fpgaux5e02ux573aux9a71ux52a8ux56e0ux7d20}
\begin{enumerate}
\tightlist
\item
  \textbf{5G通信的普及}: 5G基站和网络设备对高性能FPGA的需求大幅增加。\\
\item
  \textbf{数据中心与AI加速}:
  FPGA在数据中心中用于AI加速、云计算和边缘计算,需求持续增长。\\
\item
  \textbf{汽车电子发展}: 智能驾驶和ADAS系统对FPGA的需求快速上升。\\
\item
  \textbf{国产化替代}:
  国内厂商在FPGA领域的技术进步和国产化政策推动市场增长。
\end{enumerate}
\end{block}

\begin{block}{\textbf{CPLD市场驱动因素}}
\phantomsection\label{cpldux5e02ux573aux9a71ux52a8ux56e0ux7d20}
\begin{enumerate}
\tightlist
\item
  \textbf{工业控制需求}:
  CPLD在工业自动化、逻辑控制和接口转换中的应用稳定增长。\\
\item
  \textbf{消费电子升级}:
  显示驱动、电源管理等场景对CPLD的需求保持稳定。\\
\item
  \textbf{低成本优势}: CPLD相比FPGA成本更低,适合中小规模逻辑设计。
\end{enumerate}
\end{block}
\end{frame}

\begin{frame}
\begin{block}{国外FPGA厂家介绍}
\phantomsection\label{ux56fdux5916fpgaux5382ux5bb6ux4ecbux7ecd}
在全球FPGA(现场可编程门阵列)市场中,赛灵思(Xilinx)、\textbf{英特尔(Intel)和莱迪思半导体(Lattice
Semiconductor)}是三家最具影响力的厂商。它们凭借各自的技术优势和产品布局,占据了FPGA市场的主要份额。
\end{block}
\end{frame}

\begin{frame}
\begin{block}{赛灵思(Xilinx)}
\phantomsection\label{ux8d5bux7075ux601dxilinx}
公司简介:赛灵思成立于1984年,是FPGA技术的开创者,也是全球FPGA市场的领导者。2022年,赛灵思被AMD收购,进一步增强了其在高性能计算领域的竞争力。
主要产品:

\begin{itemize}
\tightlist
\item
  Virtex系列:面向高性能计算、数据中心和通信领域,提供高逻辑密度和强大计算能力。
\item
  Kintex系列:平衡性能和功耗,适用于工业自动化、医疗设备和视频处理等中端市场。
\item
  Artix系列:低功耗、低成本,适合消费电子和嵌入式应用。
\item
  Zynq系列:将FPGA与ARM处理器集成,广泛应用于嵌入式系统和物联网设备。
\end{itemize}

市场占有率:赛灵思长期占据FPGA市场的领先地位,2021年市场占有率约为50\%左右。
\end{block}
\end{frame}

\begin{frame}
\begin{block}{英特尔(Intel)}
\phantomsection\label{ux82f1ux7279ux5c14intel}
公司简介:英特尔通过2015年收购阿尔特拉(Altera)进入FPGA市场,成为赛灵思的主要竞争对手。英特尔将FPGA技术与其处理器产品线结合,推动其在数据中心和人工智能领域的发展。
主要产品:

\begin{itemize}
\tightlist
\item
  Stratix系列:高性能FPGA,面向数据中心加速、5G通信和军事应用。
\item
  Arria系列:中端FPGA,适用于视频处理、工业自动化和汽车电子。
\item
  Cyclone系列:低成本、低功耗FPGA,适合消费电子和物联网设备。
\item
  Agilex系列:英特尔最新一代FPGA,采用10nm工艺,支持AI加速和高性能计算。
\end{itemize}

市场占有率:英特尔(Altera)在FPGA市场的占有率约为35\%,是赛灵思的主要竞争对手。
\end{block}
\end{frame}

\begin{frame}
\begin{block}{莱迪思半导体(Lattice Semiconductor)}
\phantomsection\label{ux83b1ux8feaux601dux534aux5bfcux4f53lattice-semiconductor}
公司简介:莱迪思半导体成立于1983年,专注于低功耗、小尺寸FPGA市场,主要服务于消费电子、工业和通信领域。
主要产品:

\begin{itemize}
\tightlist
\item
  iCE系列:超低功耗FPGA,适用于移动设备、物联网和可穿戴设备。
\item
  ECP系列:低成本FPGA,面向工业自动化和消费电子。
\item
  CrossLink系列:专为视频桥接和传感器接口设计,广泛应用于汽车和工业领域。
\item
  MachXO系列:小尺寸FPGA,适合嵌入式系统和通信设备。
\end{itemize}

市场占有率:莱迪思在FPGA市场的占有率约为5\%-7\%,主要在中低端市场占据一席之地。
市场格局
赛灵思和英特尔:这两家公司占据了FPGA市场约85\%的份额,主要竞争集中在高性能FPGA领域,尤其是在数据中心、5G通信和人工智能等高端应用场景。
莱迪思半导体:专注于低功耗和小尺寸FPGA市场,凭借差异化的产品策略在特定领域保持竞争力。
其他厂商:包括Microchip(收购Microsemi)和Achronix等,市场份额较小,但在特定应用领域也有一定影响力。
\end{block}
\end{frame}

\begin{frame}
FPGA市场呈现高度集中的特点,赛灵思和英特尔作为行业双雄,主导了高性能FPGA领域;而莱迪思半导体则通过差异化定位,在低功耗和小尺寸市场占据一席之地。随着人工智能、5G通信和数据中心等新兴领域的快速发展,FPGA市场预计将继续保持增长,竞争也将更加激烈。
\end{frame}

\begin{frame}{国内FPGA厂家介绍}
\phantomsection\label{ux56fdux5185fpgaux5382ux5bb6ux4ecbux7ecd}
\begin{block}{1. \textbf{紫光同创 (Pango Micro)}}
\phantomsection\label{ux7d2bux5149ux540cux521b-pango-micro}
\begin{itemize}
\tightlist
\item
  \textbf{简介}:
  紫光同创是紫光集团旗下的FPGA设计公司,专注于高性能FPGA芯片的研发与生-
  \textbf{产品}:

  \begin{itemize}
  \tightlist
  \item
    Logos系列:中低端FPGA,适用于消费电子、工业控制等领域。\\
  \item
    Titan系列:高端FPGA,面向通信、数据中心等高性能场景。\\
  \end{itemize}
\item
  \textbf{优势}: 国产化程度高,性价比优势明显,生态逐步完善。
\end{itemize}
\end{block}
\end{frame}

\begin{frame}
\begin{block}{2. \textbf{复旦微电子 (Fudan Microelectronics)}}
\phantomsection\label{ux590dux65e6ux5faeux7535ux5b50-fudan-microelectronics}
\begin{itemize}
\tightlist
\item
  \textbf{简介}:
  复旦微电子是国内领先的集成电路设计企业,FPGA是其重要业务之一。
\item
  \textbf{产品}:

  \begin{itemize}
  \tightlist
  \item
    FMQL系列:基于ARM+FPGA架构的SoC芯片,适用于嵌入式系统。\\
  \item
    FPGA系列:覆盖中低端市场,应用于工业控制、医疗设备等领域。\\
  \end{itemize}
\item
  \textbf{优势}: 技术积累深厚,产品线丰富,支持国产化替代。
\end{itemize}
\end{block}
\end{frame}

\begin{frame}
\begin{block}{3. \textbf{安路科技 (Anlogic)}}
\phantomsection\label{ux5b89ux8defux79d1ux6280-anlogic}
\begin{itemize}
\tightlist
\item
  \textbf{简介}:
  安路科技是一家专注于FPGA芯片设计的高科技企业,致力于提供高性能、低功耗的FPGA解决方案。
\item
  \textbf{产品}:

  \begin{itemize}
  \tightlist
  \item
    Eagle系列:低功耗FPGA,适用于物联网、可穿戴设备等场景。\\
  \item
    Phoenix系列:高性能FPGA,面向通信、视频处理等领域。\\
  \end{itemize}
\item
  \textbf{优势}: 低功耗设计突出,产品性能稳定,生态逐步完善。
\end{itemize}
\end{block}
\end{frame}

\begin{frame}
\begin{block}{4. \textbf{高云半导体 (Gowin Semiconductor)}}
\phantomsection\label{ux9ad8ux4e91ux534aux5bfcux4f53-gowin-semiconductor}
\begin{itemize}
\tightlist
\item
  \textbf{简介}:
  高云半导体是国内新兴的FPGA设计公司,专注于中小容量FPGA市场。
\item
  \textbf{产品}:

  \begin{itemize}
  \tightlist
  \item
    LittleBee系列:低成本FPGA,适用于消费电子、工业控制等领域。\\
  \item
    Arora系列:中端FPGA,面向通信、视频处理等场景。\\
  \end{itemize}
\item
  \textbf{优势}: 产品性价比高,开发工具易用,生态逐步完善。
\end{itemize}
\end{block}
\end{frame}

\begin{frame}
国内FPGA厂家近年来发展迅速,产品覆盖从低端到高端的多个市场,逐步缩小与国际巨头的差距。随着国产化替代需求的增加,这些厂家在技术研发、生态建设等方面不断取得突破,未来有望在全球FPGA市场中占据更重要的地位。
\end{frame}

\begin{frame}{\textbf{FPGA与CPLD的未来趋势}}
\phantomsection\label{fpgaux4e0ecpldux7684ux672aux6765ux8d8bux52bf}
\begin{block}{\textbf{FPGA未来趋势}}
\phantomsection\label{fpgaux672aux6765ux8d8bux52bf}
\begin{enumerate}
\tightlist
\item
  \textbf{高性能与低功耗结合}:
  FPGA将向更高性能、更低功耗的方向发展,满足数据中心和AI应用的需求。\\
\item
  \textbf{异构计算}:
  FPGA与CPU、GPU的协同计算将成为主流,推动FPGA在异构计算中的应用。\\
\item
  \textbf{国产化加速}:
  国内FPGA厂商将逐步扩大市场份额,推动国产化替代进程。
\end{enumerate}
\end{block}

\begin{block}{\textbf{CPLD未来趋势}}
\phantomsection\label{cpldux672aux6765ux8d8bux52bf}
\begin{enumerate}
\tightlist
\item
  \textbf{小型化与集成化}:
  CPLD将向更小型化、更高集成度的方向发展,满足消费电子和工业控制的需求。\\
\item
  \textbf{低成本解决方案}:
  CPLD将继续作为低成本、低功耗的逻辑解决方案,在特定领域保持竞争力。\\
\item
  \textbf{新兴应用拓展}:
  CPLD在物联网、智能家居等新兴领域的应用有望逐步增加。
\end{enumerate}
\end{block}
\end{frame}

\end{document}
