\documentclass{beamer}

% 使用 xelatex 编译,引入 xeCJK 包支持中文
\usepackage[UTF8]{ctex}
\usepackage{fontspec}
\setmainfont{SimSun} % 设置中文字体为宋体
\usepackage{tabularx}
\usepackage{tikz}


% 主题设置
\usetheme{Dresden}
\usecolortheme{default}

\providecommand{\tightlist}{%
  \setlength{\itemsep}{0pt}\setlength{\parskip}{0pt}}

% 标题页信息
\title{FPGA系统设计}
\subtitle{介绍}
\author{周贤中}
\institute{广东工业大学集成电路学院}
\date{\today}

% 文档内容
\begin{document}

% 标题页
\begin{frame}
\titlepage
\end{frame}

% 目录页
\begin{frame}
\frametitle{目录}
\tableofcontents
\end{frame}

\section{PLD与ASIC的基本概念}
\subsection{PLD的基本概念}
\begin{frame}{\textbf{PLD的基本概念}}
可编程逻辑器件(Programmable Logic Device,
PLD)是一类可以通过编程实现特定逻辑功能的集成电路。PLD的主要特点是通过软件工具对硬件进行配置,从而实现不同的逻辑功能。以下是PLD的核心概念:
\end{frame}

\begin{frame}{\textbf{PLD的分类}}
PLD根据其复杂性和功能可以分为以下几类:

\begin{enumerate}
\tightlist
\item
    \textbf{SPLD(简单可编程逻辑器件)}:

    \begin{itemize}
    \tightlist
    \item
    包括PAL(可编程阵列逻辑)和GAL(通用阵列逻辑)。\\
    \item
    适合实现简单的逻辑功能,如组合逻辑和时序逻辑。\\
    \end{itemize}
\item
    \textbf{CPLD(复杂可编程逻辑器件)}:

    \begin{itemize}
    \tightlist
    \item
    由多个SPLD模块和可编程互连资源组成。\\
    \item
    适合实现中等复杂度的逻辑功能,如状态机和接口转换。\\
    \end{itemize}
\item
    \textbf{FPGA(现场可编程门阵列)}:

    \begin{itemize}
    \tightlist
    \item
    由大量可配置逻辑块(CLB)、可编程互连资源和I/O单元组成。\\
    \item
    适合实现复杂的逻辑功能,如数字信号处理和通信协议。
    \end{itemize}
\end{enumerate}
\end{frame}

\begin{frame}{\textbf{PLD的特点}}
\begin{itemize}
\tightlist
\item
    \textbf{可编程性}: 用户可以通过编程实现不同的逻辑功能。\\
\item
    \textbf{灵活性}: 支持动态重构,适合快速原型设计和迭代开发。\\
\item
    \textbf{开发周期短}: 相比ASIC,PLD的开发周期显著缩短。\\
\item
    \textbf{成本适中}: 适合中小批量生产,无需高昂的流片成本。
\end{itemize}
\end{frame}

\begin{frame}{\textbf{PLD的应用场景}}
\begin{itemize}
\tightlist
\item
    通信设备(如5G基站、网络交换机)\\
\item
    工业控制(如PLC、机器人控制)\\
\item
    消费电子(如视频处理、游戏硬件)\\
\item
    原型设计与验证(如ASIC设计的前期验证)
\end{itemize}
\end{frame}

\subsection{ASIC的基本概念}

\begin{frame}{\textbf{ASIC的基本概念}}
专用集成电路(Application-Specific Integrated Circuit,
ASIC)是为特定应用定制的集成电路,设计完成后功能固定,无法更改。以下是ASIC的核心概念:
\end{frame}

\begin{frame}{\textbf{ASIC的分类}}
\begin{enumerate}
\tightlist
\item
    \textbf{全定制ASIC}:

    \begin{itemize}
    \tightlist
    \item
    从晶体管级别进行设计,性能最优,但开发成本和时间最高。\\
    \end{itemize}
\item
    \textbf{半定制ASIC}:

    \begin{itemize}
    \tightlist
    \item
    基于标准单元库或门阵列进行设计,性能和成本介于全定制和可编程器件之间。\\
    \end{itemize}
\item
    \textbf{结构化ASIC}:

    \begin{itemize}
    \tightlist
    \item
    基于预定义的硬件结构进行设计,开发周期较短,成本较低。
    \end{itemize}
\end{enumerate}
\end{frame}

\begin{frame}{\textbf{ASIC的特点}}
\begin{itemize}
\tightlist
\item
    \textbf{高性能}: 针对特定应用优化,性能优于PLD。\\
\item
    \textbf{低功耗}: 针对特定应用进行功耗优化,功耗低于PLD。\\
\item
    \textbf{高成本}: 开发成本高,适合大批量生产。\\
\item
    \textbf{开发周期长}: 从设计到流片需要数月甚至数年时间。
\end{itemize}
\end{frame}

\begin{frame}{\textbf{ASIC的应用场景}}
\begin{itemize}
\tightlist
\item
    消费电子(如智能手机、平板电脑)\\
\item
    汽车电子(如ADAS、车载娱乐系统)\\
\item
    数据中心(如AI加速芯片、网络处理器)\\
\item
    工业设备(如传感器、控制器)
\end{itemize}
\end{frame}

\subsection{PLD与ASIC的详细对比分析}
\begin{frame}{\textbf{设计灵活性}}
\begin{itemize}
\tightlist
\item
    \textbf{PLD}:

    \begin{itemize}
    \tightlist
    \item
    高灵活性,可编程实现任意逻辑功能\\
    \item
    支持动态重构\\
    \item
    适合快速原型设计和迭代开发\\
    \item
    适用于需频繁更新的场景\\
    \end{itemize}
\item
    \textbf{ASIC}:

    \begin{itemize}
    \tightlist
    \item
    功能固化成品后无法修改\\
    \item
    适合固定功能场景\\
    \item
    设计修改需重新流片,成本高昂
    \end{itemize}
\end{itemize}
\end{frame}

\begin{frame}{\textbf{性能}}

\begin{itemize}
\tightlist
\item
    \textbf{PLD}:

    \begin{itemize}
    \tightlist
    \item
    中等复杂度设计适用\\
    \item
    信号传输延迟较高\\
    \item
    适合对性能要求不高的场景\\
    \end{itemize}
\item
    \textbf{ASIC}:

    \begin{itemize}
    \tightlist
    \item
    性能最高,针对特定场景优化\\
    \item
    信号传输延迟低\\
    \item
    适合高速通信、AI加速等高性能需求场景
    \end{itemize}
\end{itemize}
\end{frame}

\begin{frame}{\textbf{功耗}}

\begin{itemize}
\tightlist
\item
    \textbf{PLD}:

    \begin{itemize}
    \tightlist
    \item
    功耗较高(静态/动态)\\
    \item
    适合对功耗要求宽松的场景\\
    \end{itemize}
\item
    \textbf{ASIC}:

    \begin{itemize}
    \tightlist
    \item
    功耗最低,经定制优化\\
    \item
    适合移动设备、物联网等低功耗场景
    \end{itemize}
\end{itemize}
\end{frame}

\begin{frame}{\textbf{成本}}

\begin{itemize}
\tightlist
\item
    \textbf{PLD}:

    \begin{itemize}
    \tightlist
    \item
    中等成本,中小批量生产适用\\
    \item
    开发成本低,无流片费用\\
    \end{itemize}
\item
    \textbf{ASIC}:

    \begin{itemize}
    \tightlist
    \item
    大批量时单位成本低\\
    \item
    流片成本高昂\\
    \item
    适合大规模生产分摊成本
    \end{itemize}
\end{itemize}
\end{frame}

\begin{frame}{\textbf{开发周期}}

\begin{itemize}
\tightlist
\item
    \textbf{PLD}:

    \begin{itemize}
    \tightlist
    \item
    周期短(数周至数月)\\
    \item
    设计流程简单\\
    \item
    适合时间紧迫项目\\
    \end{itemize}
\item
    \textbf{ASIC}:

    \begin{itemize}
    \tightlist
    \item
    周期长(数月至数年)\\
    \item
    需复杂物理设计和验证流程\\
    \item
    适合长期高要求项目
    \end{itemize}
\end{itemize}
\end{frame}

\begin{frame}{\textbf{开发工具}}

\begin{itemize}
\tightlist
\item
    \textbf{PLD}:

    \begin{itemize}
    \tightlist
    \item
    需专用EDA工具(如Vivado, Quartus)\\
    \item
    工具链成熟,快速验证\\
    \item
    适合PLD工程师\\
    \end{itemize}
\item
    \textbf{ASIC}:

    \begin{itemize}
    \tightlist
    \item
    需ASIC工具链(如Cadence, Synopsys)\\
    \item
    工具复杂,需专业团队\\
    \item
    适用经验丰富的ASIC工程师
    \end{itemize}
\end{itemize}
\end{frame}

\begin{frame}{\textbf{适用场景}}

\begin{itemize}
\tightlist
\item
    \textbf{PLD}:

    \begin{itemize}
    \tightlist
    \item
    通信设备(5G基站、交换机)\\
    \item
    工业控制(PLC、机器人)\\
    \item
    消费电子(视频处理)\\
    \item
    ASIC原型验证\\
    \end{itemize}
\item
    \textbf{ASIC}:

    \begin{itemize}
    \tightlist
    \item
    消费电子(手机、平板)\\
    \item
    汽车电子(ADAS)\\
    \item
    数据中心(AI加速器)\\
    \item
    工业设备(传感器)
    \end{itemize}
\end{itemize}
\end{frame}

\begin{frame}{\textbf{量产成本}}

\begin{itemize}
\tightlist
\item
    \textbf{PLD}:

    \begin{itemize}
    \tightlist
    \item
    中小批量成本较高\\
    \item
    单位成本随规模降低\\
    \end{itemize}
\item
    \textbf{ASIC}:

    \begin{itemize}
    \tightlist
    \item
    大批量单位成本极低\\
    \item
    适合大规模生产
    \end{itemize}
\end{itemize}
\end{frame}

\begin{frame}{\textbf{可编程性}}

\begin{itemize}
\tightlist
\item
    \textbf{PLD}:

    \begin{itemize}
    \tightlist
    \item
    可重复编程和动态重构\\
    \item
    支持功能灵活调整\\
    \end{itemize}
\item
    \textbf{ASIC}:

    \begin{itemize}
    \tightlist
    \item
    功能固化成形后不可修改
    \end{itemize}
\end{itemize}
\end{frame}

\begin{frame}{\textbf{集成度}}

\begin{itemize}
\tightlist
\item
    \textbf{PLD}:

    \begin{itemize}
    \tightlist
    \item
    高集成度(数百万逻辑单元)\\
    \item
    支持复杂算法和协议\\
    \end{itemize}
\item
    \textbf{ASIC}:

    \begin{itemize}
    \tightlist
    \item
    定制化集成更多模块\\
    \item
    优化集成度和性能
    \end{itemize}
\end{itemize}

\textbf{设计复杂度}

\begin{itemize}
\tightlist
\item
    \textbf{PLD}:

    \begin{itemize}
    \tightlist
    \item
    中等复杂度,适用中小规模设计\\
    \item
    工具提供丰富IP核\\
    \end{itemize}
\item
    \textbf{ASIC}:

    \begin{itemize}
    \tightlist
    \item
    复杂度高,需完整流程(逻辑设计→物理设计→流片)\\
    \item
    需大型专业团队
    \end{itemize}
\end{itemize}
\end{frame}

\begin{frame}{\textbf{风险与可靠性}}

\begin{itemize}
\tightlist
\item
    \textbf{PLD}:

    \begin{itemize}
    \tightlist
    \item
    风险低(可重新编程修复)\\
    \item
    适合原型验证\\
    \item
    可靠性依赖器件寿命\\
    \end{itemize}
\item
    \textbf{ASIC}:

    \begin{itemize}
    \tightlist
    \item
    风险高(需重新流片修复错误)\\
    \item
    需充分验证设计\\
    \item
    高可靠性(经应用优化)
    \end{itemize}
\end{itemize}
\end{frame}

\begin{frame}{\textbf{生态系统}}

\begin{itemize}
\tightlist
\item
    \textbf{PLD}:

    \begin{itemize}
    \tightlist
    \item
    成熟生态(工具、IP核、社区)\\
    \item
    厂商支持完善(Xilinx/Intel)\\
    \item
    适合中小企业和初创\\
    \end{itemize}
\item
    \textbf{ASIC}:

    \begin{itemize}
    \tightlist
    \item
    需专业团队和代工厂(TSMC/三星)\\
    \item
    工具链复杂,成本高昂\\
    \item
    适合大企业或资金充足项目
    \end{itemize}
\end{itemize}
\end{frame}

\begin{frame}[allowframebreaks]
\frametitle{PLD vs ASIC 核心对比}
\begin{block}{特性对比}
% \tiny
\begin{tabular}{|p{1.8cm}|p{4cm}|p{4cm}|}
\hline
\textbf{维度} & \textbf{PLD} & \textbf{ASIC} \\ \hline
设计特征 & 
$\bullet$ 可重构\newline
$\bullet$ 性能/功耗中等\newline
$\bullet$ 开发周期:周级
& 
$\bullet$ 功能固化\newline
$\bullet$ 高性能/低功耗\newline
$\bullet$ 开发周期:月/年级 \\ \hline

生态系统 & 
$\bullet$ Xilinx/Intel工具\newline
$\bullet$ 开源社区\newline
$\bullet$ 云开发
& 
$\bullet$ TSMC/三星代工\newline
$\bullet$ EDA厂商协作\newline
$\bullet$ 专业IP供应商 \\ \hline

经济模型 & 
$\bullet$ 开发成本:\$10$^3$-\$10$^5$\newline
$\bullet$ 中小批量
& 
$\bullet$ NRE成本:\$10$^6$+\newline
$\bullet$ 量产成本:\$10$^0$/unit \\ \hline

风险控制 & 
$\bullet$ 动态重编程\newline
$\bullet$ 多版本调试
& 
$\bullet$ FIB修改:\$10$^5$/次\newline
$\bullet$ 金属掩模费用 \\ \hline

典型应用 & 
$\bullet$ 5G基站\newline
$\bullet$ 工控系统
& 
$\bullet$ 手机SoC\newline
$\bullet$ AI加速 \\ \hline
\end{tabular}
\end{block}
\pagebreak
\begin{block}{决策矩阵}
% \tiny
\begin{tabular}{|p{1.8cm}|p{4cm}|p{4cm}|}
\hline
\textbf{条件} & \textbf{选PLD} & \textbf{选ASIC} \\ \hline
项目规模 & 团队<20人\newline 预算<\$1M & 团队>50人\newline 预算>\$10M \\ 
迭代需求 & 功能未定型\newline >3次/年 & 架构冻结\newline <1次/3年 \\
芯片用量 & <10k片/年 & >1M片/年 \\
认证要求 & 商用级验证 & 车规级/军工级认证 \\
\hline
\end{tabular}
\end{block}
\end{frame}

\begin{frame}[allowframebreaks]{典型案例}
\begin{block}{PLD的典型应用}
\begin{itemize}
\tightlist
\item
    通信设备:
    5G基站、网络交换机等设备中,FPGA用于实现高速信号处理和协议转换。
\item
    工业控制:
    PLC(可编程逻辑控制器)和机器人控制中,CPLD用于实现逻辑控制和接口转换。
\item
    消费电子: 视频处理、游戏硬件等设备中,FPGA用于实现图像处理和算法加速。
\item
    原型设计与验证: 在ASIC设计的前期,FPGA用于功能验证和性能测试。
\end{itemize}
\end{block}

\begin{block}{ASIC的典型应用}
\begin{itemize}
\tightlist
\item
    消费电子:
    智能手机、平板电脑等设备中,ASIC用于实现处理器、基带芯片和图像传感器。
\item
    汽车电子:
    ADAS(高级驾驶辅助系统)和车载娱乐系统中,ASIC用于实现传感器处理和控制算法。
\item
    数据中心: AI加速芯片和网络处理器中,ASIC用于实现高性能计算和数据传输。
\item
    工业设备: 传感器和控制器中,ASIC用于实现高精度测量和控制逻辑。
\end{itemize}
\end{block}
\end{frame}

\section{FPGA市场分析}

\begin{frame}{\textbf{FPGA与CPLD的市场份额对比}}
以下是FPGA与CPLD在全球市场中的份额对比:

在全球可编程逻辑器件市场中,FPGA和CPLD的市场表现存在明显差异:

\begin{itemize}
\tightlist
\item
    \textbf{FPGA}:

    \begin{itemize}
    \tightlist
    \item
    2022年市场规模:80亿美元
    \item
    主要厂商:赛灵思(现隶属于AMD)、英特尔(Altera系列)、紫光同创、复旦微电子
    \end{itemize}
\item
    \textbf{CPLD}:

    \begin{itemize}
    \tightlist
    \item
    2022年市场规模:10亿美元
    \item
    主要厂商:莱迪思半导体、微芯科技
    \end{itemize}
\end{itemize}
\end{frame}

\begin{frame}[allowframebreaks]{\textbf{FPGA与CPLD的市场驱动因素}}
\begin{block}{\textbf{FPGA市场驱动因素}}
\begin{enumerate}
\tightlist
\item
    \textbf{5G通信的普及}: 5G基站和网络设备对高性能FPGA的需求大幅增加。\\
\item
    \textbf{数据中心与AI加速}:
    FPGA在数据中心中用于AI加速、云计算和边缘计算,需求持续增长。\\
\item
    \textbf{汽车电子发展}: 智能驾驶和ADAS系统对FPGA的需求快速上升。\\
\item
    \textbf{国产化替代}:
    国内厂商在FPGA领域的技术进步和国产化政策推动市场增长。
\end{enumerate}
\end{block}
\pagebreak
\begin{block}{\textbf{CPLD市场驱动因素}}
\begin{enumerate}
\tightlist
\item
    \textbf{工业控制需求}:
    CPLD在工业自动化、逻辑控制和接口转换中的应用稳定增长。\\
\item
    \textbf{消费电子升级}:
    显示驱动、电源管理等场景对CPLD的需求保持稳定。\\
\item
    \textbf{低成本优势}: CPLD相比FPGA成本更低,适合中小规模逻辑设计。
\end{enumerate}
\end{block}
\end{frame}

\begin{frame}
\begin{block}{国外FPGA厂家介绍}
在全球FPGA(现场可编程门阵列)市场中,赛灵思(Xilinx)、\textbf{英特尔(Intel)和莱迪思半导体(Lattice
Semiconductor)}是三家最具影响力的厂商。它们凭借各自的技术优势和产品布局,占据了FPGA市场的主要份额。
\end{block}
\end{frame}

\begin{frame}{赛灵思(Xilinx)}
公司简介:赛灵思成立于1984年,是FPGA技术的开创者,也是全球FPGA市场的领导者。2022年,赛灵思被AMD收购,进一步增强了其在高性能计算领域的竞争力。
主要产品:

\begin{itemize}
\tightlist
\item
    Virtex系列:面向高性能计算、数据中心和通信领域,提供高逻辑密度和强大计算能力。
\item
    Kintex系列:平衡性能和功耗,适用于工业自动化、医疗设备和视频处理等中端市场。
\item
    Artix系列:低功耗、低成本,适合消费电子和嵌入式应用。
\item
    Zynq系列:将FPGA与ARM处理器集成,广泛应用于嵌入式系统和物联网设备。
\end{itemize}

市场占有率:赛灵思长期占据FPGA市场的领先地位,2021年市场占有率约为50\%左右。
\end{frame}

\begin{frame}{英特尔(Intel)}
公司简介:英特尔通过2015年收购阿尔特拉(Altera)进入FPGA市场,成为赛灵思的主要竞争对手。英特尔将FPGA技术与其处理器产品线结合,推动其在数据中心和人工智能领域的发展。
主要产品:

\begin{itemize}
\tightlist
\item
    Stratix系列:高性能FPGA,面向数据中心加速、5G通信和军事应用。
\item
    Arria系列:中端FPGA,适用于视频处理、工业自动化和汽车电子。
\item
    Cyclone系列:低成本、低功耗FPGA,适合消费电子和物联网设备。
\item
    Agilex系列:英特尔最新一代FPGA,采用10nm工艺,支持AI加速和高性能计算。
\end{itemize}

市场占有率:英特尔(Altera)在FPGA市场的占有率约为35\%,是赛灵思的主要竞争对手。
\end{frame}

\begin{frame}{莱迪思半导体(Lattice Semiconductor)}
公司简介:莱迪思半导体成立于1983年,专注于低功耗、小尺寸FPGA市场,主要服务于消费电子、工业和通信领域。
主要产品:

\begin{itemize}
\tightlist
\item
    iCE系列:超低功耗FPGA,适用于移动设备、物联网和可穿戴设备。
\item
    ECP系列:低成本FPGA,面向工业自动化和消费电子。
\item
    CrossLink系列:专为视频桥接和传感器接口设计,广泛应用于汽车和工业领域。
\item
    MachXO系列:小尺寸FPGA,适合嵌入式系统和通信设备。
\end{itemize}

市场占有率:莱迪思在FPGA市场的占有率约为5\%-7\%,主要在中低端市场占据一席之地。
\end{frame}

\begin{frame}{\textbf{FPGA市场竞争格局}}
\begin{columns}[T]
\column{0.48\textwidth}
\begin{block}{\textbf{市场特征}}
\begin{itemize}
\scriptsize
\item \textbf{双头垄断结构}: 赛灵思(Xilinx)和英特尔(Altera)合计市占率达85\%
\item \textbf{应用驱动}: AI/5G/数据中心催生高端FPGA需求 
\item \textbf{年增长率}: 维持8\% CAGR至2027年
\end{itemize}
\end{block}

\column{0.48\textwidth}
\begin{block}{\textbf{竞争版图}}
% \includegraphics[width=\textwidth]{market_share_chart.pdf}  % 建议补充柱状图示意
\begin{itemize}
\tiny
\item 赛灵思:数据中心主导者(49\%)
\item Intel:5G基础设施领跑(36\%)
\end{itemize}
\end{block}
\end{columns}

\begin{block}{\textbf{厂商动态}}
\begin{itemize}
\footnotesize
\item \textcolor{blue}{\textbf{头部厂商}}: 
  - 赛灵思(AMD)重点布局AI推理加速
  - 英特尔聚焦5G基站优化方案
  
\item \textcolor{green}{\textbf{差异化竞争}}: 
  - 莱迪思半导体(Lattice)专注低功耗FPGA
  - 产品尺寸缩小40\%,功耗降低30\%

\item \textcolor{orange}{\textbf{新兴力量}}: 
  - Achronix新型FPGA架构Speedster7t
  - 国产厂商突破28nm工艺
\end{itemize}
\end{block}
\end{frame}

\begin{frame}[allowframebreaks]{国内FPGA厂家介绍}
\begin{block}{1. \textbf{紫光同创 (Pango Micro)}}
\begin{itemize}
\tightlist
\item
    \textbf{简介}:
    紫光同创是紫光集团旗下的FPGA设计公司,专注于高性能FPGA芯片的研发与生-
    \textbf{产品}:

    \begin{itemize}
    \tightlist
    \item
    Logos系列:中低端FPGA,适用于消费电子、工业控制等领域。\\
    \item
    Titan系列:高端FPGA,面向通信、数据中心等高性能场景。\\
    \end{itemize}
\item
    \textbf{优势}: 国产化程度高,性价比优势明显,生态逐步完善。
\end{itemize}
\end{block}
\pagebreak
\begin{block}{2. \textbf{复旦微电子 (Fudan Microelectronics)}}
\begin{itemize}
\tightlist
\item
    \textbf{简介}:
    复旦微电子是国内领先的集成电路设计企业,FPGA是其重要业务之一。
\item
    \textbf{产品}:

    \begin{itemize}
    \tightlist
    \item
    FMQL系列:基于ARM+FPGA架构的SoC芯片,适用于嵌入式系统。\\
    \item
    FPGA系列:覆盖中低端市场,应用于工业控制、医疗设备等领域。\\
    \end{itemize}
\item
    \textbf{优势}: 技术积累深厚,产品线丰富,支持国产化替代。
\end{itemize}
\end{block}
\pagebreak
\begin{block}{3. \textbf{安路科技 (Anlogic)}}
\begin{itemize}
\tightlist
\item
    \textbf{简介}:
    安路科技是一家专注于FPGA芯片设计的高科技企业,致力于提供高性能、低功耗的FPGA解决方案。
\item
    \textbf{产品}:

    \begin{itemize}
    \tightlist
    \item
    Eagle系列:低功耗FPGA,适用于物联网、可穿戴设备等场景。\\
    \item
    Phoenix系列:高性能FPGA,面向通信、视频处理等领域。\\
    \end{itemize}
\item
    \textbf{优势}: 低功耗设计突出,产品性能稳定,生态逐步完善。
\end{itemize}
\end{block}
\pagebreak
\begin{block}{4. \textbf{高云半导体 (Gowin Semiconductor)}}
\begin{itemize}
\tightlist
\item
    \textbf{简介}:
    高云半导体是国内新兴的FPGA设计公司,专注于中小容量FPGA市场。
\item
    \textbf{产品}:

    \begin{itemize}
    \tightlist
    \item
    LittleBee系列:低成本FPGA,适用于消费电子、工业控制等领域。\\
    \item
    Arora系列:中端FPGA,面向通信、视频处理等场景。\\
    \end{itemize}
\item
    \textbf{优势}: 产品性价比高,开发工具易用,生态逐步完善。
\end{itemize}
\end{block}
\pagebreak
国内FPGA厂家近年来发展迅速,产品覆盖从低端到高端的多个市场,逐步缩小与国际巨头的差距。随着国产化替代需求的增加,这些厂家在技术研发、生态建设等方面不断取得突破,未来有望在全球FPGA市场中占据更重要的地位。
\end{frame}

\section{FPGA与CPLD的未来趋势}
\begin{frame}[allowframebreaks]{\textbf{FPGA与CPLD的未来趋势}}
\begin{block}{\textbf{FPGA未来趋势}}
\begin{enumerate}
\tightlist
\item
    \textbf{高性能与低功耗结合}:
    FPGA将向更高性能、更低功耗的方向发展,满足数据中心和AI应用的需求。\\
\item
    \textbf{异构计算}:
    FPGA与CPU、GPU的协同计算将成为主流,推动FPGA在异构计算中的应用。\\
\item
    \textbf{国产化加速}:
    国内FPGA厂商将逐步扩大市场份额,推动国产化替代进程。
\end{enumerate}
\end{block}

\begin{block}{\textbf{CPLD未来趋势}}
\begin{enumerate}
\tightlist
\item
    \textbf{小型化与集成化}:
    CPLD将向更小型化、更高集成度的方向发展,满足消费电子和工业控制的需求。\\
\item
    \textbf{低成本解决方案}:
    CPLD将继续作为低成本、低功耗的逻辑解决方案,在特定领域保持竞争力。\\
\item
    \textbf{新兴应用拓展}:
    CPLD在物联网、智能家居等新兴领域的应用有望逐步增加。
\end{enumerate}
\end{block}
\end{frame}
        

% 结尾页
\begin{frame}
\centering
\frametitle{谢谢!}
感谢聆听!
\end{frame}

\end{document}
